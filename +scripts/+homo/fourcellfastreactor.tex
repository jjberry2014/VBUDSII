
\documentclass[letterpaper,12pt]{article}
\usepackage{graphicx}
\usepackage[left=1in,right=1in,top=1in,bottom=1in]{geometry}
\usepackage{caption}
\usepackage{times}
\usepackage{natbib}
%\usepackage[
%    pdfborder={0 0 0}, 
%    colorlinks={true},
%    citecolor=red,
%    linkbordercolor={0 0 0},
%    linkcolor=blue
%    ]{hyperref}
\newcommand{\fitzefigtwo}[2]{
    \begin{center}
    \includegraphics{#1}
    \captionof{figure}{#2}
    \label{fig:#1}
    \end{center}
    }
\newcommand{\fitzetabtwo}[2]{
    \begin{center}
    \includegraphics{#1}
    \captionof{table}{#2}
    \label{tab:#1}
    \end{center}
    }
\newcommand{\fitzetabthree}[3]{
    \begin{center}
    \includegraphics[width=#2]{#1}
    \captionof{table}{#3}
    \label{tab:#1}
    \end{center}
    }

\begin{document}

\begin{center}\textbf{Four cell fast reactor results.}\end{center}

\fitzetabtwo{multifastgeoffinput}{Dimensions and composition of a fast reactor
fueled with both uranium oxide and plutonium oxide and cooled with sodium. The
fuel rod is modeled by smearing steel throughout the sodium. This reactor
definition is informed by the MOX reactor in \citep{PoPRV4}.}

\fitzefigtwo{multifastgeoffuo2flux}{Comparison to MCNPX of spectral flux in the
UOX fuel. The top graph compares the fuel flux from our method (``CPM'') to
MCNPX results. The bottom graph provides the error between our method and
MCNPX. The coefficient of determination between the results is 0.9966.}

\fitzefigtwo{multifastgeoffplutflux}{Comparison to MCNPX of spectral flux in
the plutonium fuel. The top graph compares the fuel flux from our method
(``CPM'') to MCNPX results. The bottom graph provides the error between our
method and MCNPX. The coefficient of determination between the results is
0.9966.}

\fitzetabtwo{multireactionrates1g}{Comparison to
MCNPX of one-group reaction rates in a sodium fast reactor. The absorption
rates for the fuel isotopes are provided. Our method is noted as ``CPM''.
These reaction rates cover the energy range from
5 eV to 10 MeV. The numbers are normalized to 100,000 total absorptions in
the system.}

\bibliographystyle{elsarticle-harv}
\bibliography{library.bib}

\end{document}
