\documentclass[letterpaper,12pt]{article}
\usepackage[left=1in,right=1in,top=1in,bottom=1in]{geometry}
\usepackage{amsmath}
\newcommand{\pito}[2]{\Pi^{#1 \gets #2}}

\begin{document}
\title{VBUDSII: The second "individual" homogenization approach.}
\author{Chris Dembia}
\date{\today}

\maketitle

This document presents the formulation for the second homogenization approach
for VBUDSII. This method has been developed because the original formulation
described in the MS thesis by Chris Dembia does not yield sufficiently
accurate results. The first homogenization formulation is described in Dembia's
thesis.

The method is presented for a four-cell implementation. The four cells are:

\begin{enumerate}
\item AA: cell pair A, antipin (e.g. UOX fuel).
\item AP: cell pair A, pin (e.g. UOX's coolant).
\item BA: cell pair B, anitpin (e.g. IMF fuel).
\item BP: cell pair B, pin (e.g. IMF's coolant).
\end{enumerate}

We obtain a mean free path in each cell as a function of energy group:

\begin{align}
\lambda^{AA}(E) = 1/\Sigma_{t}^{AA}(E) \\
\lambda^{AP}(E) = 1/\Sigma_{t}^{AP}(E) \\
\lambda^{BA}(E) = 1/\Sigma_{t}^{BA}(E) \\
\lambda^{BP}(E) = 1/\Sigma_{t}^{BP}(E)
\end{align}

We use these mean free paths to obtain the homogenization parameter in each
cell for each energy group.

\begin{align}
    h^{AA} = h(\lambda^{AA}(E)) \\
    h^{AP} = h(\lambda^{AP}(E))  \\
    h^{BA} = h(\lambda^{BA}(E))  \\
    h^{BP} = h(\lambda^{BP}(E)) 
\end{align}

where $h(\lambda)$ is given by:

\begin{equation}
    h(\lambda) = \frac{1}{2}\left(1 + \mbox{erf}{[k(\lambda -
    \lambda^h)]}\right)
\end{equation}

\noindent but could be given by a similar expression. The quantities $k$
and $\lambda^h$ are parameters that the user can change; $\lambda^h$ should be
set to some characteristic dimension of the cell and there is little to be
said as of yet about how $k$ should be chosen. The value of $h$ is always less
than or equal to 1.

In each column of the $\mathbf{\Pi}$ matrix, two terms represent probabilities of
remaining in the cell pair, and two represent probabilities of moving to the
other cell pair. The latter of these two, in each column, are scaled down by
the factor $h$.

The $\mathbf{\Pi}$ matrix has the following form:

\begin{align}
    \mathbf{\Pi} = 
    \begin{bmatrix}
        \pito{AA}{AA} & \mathbf{\pito{AA}{BA}} & \pito{AA}{AP} &
        \mathbf{\pito{AA}{BP}} \\
        \mathbf{\pito{BA}{AA}} & \pito{BA}{BA} & \mathbf{\pito{BA}{AP}} & \pito{BA}{BP} \\
        \pito{AP}{AA} & \mathbf{\pito{AP}{BA}} & \pito{AP}{AP} &
        \mathbf{\pito{AP}{BP}} \\
        \mathbf{\pito{BP}{AA}} & \pito{BP}{BA} & \mathbf{\pito{BP}{AP}} & \pito{BP}{BP} \\
    \end{bmatrix}
\end{align}

The entries representing out-of-pair transport are emboldened, and it is these
quantities that are reduced by the appropriate $h$. This introduces the issue
that the columns of the matrix no longer sum to 1. This is managed by scaling the
remaining two terms in each column by a single factor that brings the sum back
t 1. For the first column, we have:

\begin{equation}
    \gamma (\pito{AA}{AA} + \pito{AP}{AA}) + h(\pito{BA}{AA} + \pito{BP}{AA}) =
    1
\end{equation}

so that $\gamma$ is given by:

\begin{equation}
    \gamma  = \frac{1 - h(\pito{BA}{AA} + \pito{BP}{AA})}{\pito{AA}{AA} +
    \pito{AP}{AA}}
\end{equation}

The final form of the $\mathbf{\Pi}$ matrix is then:

\begin{align}
    \mathbf{\Pi} = 
    \begin{bmatrix}
        \gamma^{AA}\pito{AA}{AA} & h^{BA}\mathbf{\pito{AA}{BA}} & \gamma^{AP}\pito{AA}{AP} &
        h^{BP}\mathbf{\pito{AA}{BP}} \\
        h^{AA}\mathbf{\pito{BA}{AA}} & \gamma^{BA}\pito{BA}{BA} &
        h^{AP}\mathbf{\pito{BA}{AP}} & \gamma^{BP}\pito{BA}{BP} \\
        \gamma^{AA}\pito{AP}{AA} & h^{BA}\mathbf{\pito{AP}{BA}} & \gamma^{AP}\pito{AP}{AP} &
        h^{BP}\mathbf{\pito{AP}{BP}} \\
        h^{AA}\mathbf{\pito{BP}{AA}} & \gamma^{BA}\pito{BP}{BA} &
        h^{AP}\mathbf{\pito{BP}{AP}} & \gamma^{BP}\pito{BP}{BP} \\
    \end{bmatrix}
\end{align}







\end{document}

